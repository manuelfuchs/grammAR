\firstsection{Introduction}

\maketitle

%% \section{Introduction} %for journal use above \firstsection{..} instead
This template is for papers of VGTC-sponsored conferences which are \emph{\textbf{not}} published in a special issue of TVCG.

\section{Using the Style Template}

\begin{itemize}
\item If you receive compilation errors along the lines of ``\texttt{Package ifpdf Error: Name clash, \textbackslash ifpdf is already defined}'' then please add a new line ``\texttt{\textbackslash let\textbackslash ifpdf\textbackslash relax}'' right after the ``\texttt{\textbackslash documentclass[journal]\{vgtc\}}'' call. Note that your error is due to packages you use that define ``\texttt{\textbackslash ifpdf}'' which is obsolete (the result is that \texttt{\textbackslash ifpdf} is defined twice); these packages should be changed to use ifpdf package instead.
\item The style uses the hyperref package, thus turns references into internal links. We thus recommend to make use of the ``\texttt{\textbackslash autoref\{reference\}}'' call (instead of ``\texttt{Figure\~{}\textbackslash ref\{reference\}}'' or similar) since ``\texttt{\textbackslash autoref\{reference\}}'' turns the entire reference into an internal link, not just the number. Examples: \autoref{fig:sample} and \autoref{tab:vis_papers}.
\item The style automatically looks for image files with the correct extension (eps for regular \LaTeX; pdf, png, and jpg for pdf\LaTeX), in a set of given subfolders (figures/, pictures/, images/). It is thus sufficient to use ``\texttt{\textbackslash includegraphics\{CypressView\}}'' (instead of ``\texttt{\textbackslash includegraphics\{pictures/CypressView.jpg\}}'').
\item For adding hyperlinks and DOIs to the list of references, you can use ``\texttt{\textbackslash bibliographystyle\{abbrv-doi-hyperref-narrow\}}'' (instead of ``\texttt{\textbackslash bibliographystyle\{abbrv\}}''). It uses the doi and url fields in a bib\TeX\ entry and turns the entire reference into a link, giving priority to the doi. The doi can be entered with or without the ``\texttt{http://dx.doi.org/}'' url part. See the examples in the bib\TeX\ file and the bibliography at the end of this template.\\[1em]
\textbf{Note 1:} occasionally (for some \LaTeX\ distributions) this hyper-linked bib\TeX\ style may lead to \textbf{compilation errors} (``\texttt{pdfendlink ended up in different nesting level ...}'') if a reference entry is broken across two pages (due to a bug in hyperref). In this case make sure you have the latest version of the hyperref package (i.\,e., update your \LaTeX\ installation/packages) or, alternatively, revert back to ``\texttt{\textbackslash bibliographystyle\{abbrv-doi-narrow\}}'' (at the expense of removing hyperlinks from the bibliography) and try ``\texttt{\textbackslash bibliographystyle\{abbrv-doi-hyperref-narrow\}}'' again after some more editing.\\[1em]
\textbf{Note 2:} the ``\texttt{-narrow}'' versions of the bibliography style use the font ``PTSansNarrow-TLF'' for typesetting the DOIs in a compact way. This font needs to be available on your \LaTeX\ system. It is part of the \href{https://www.ctan.org/pkg/paratype}{``paratype'' package}, and many distributions (such as MikTeX) have it automatically installed. If you do not have this package yet and want to use a ``\texttt{-narrow}'' bibliography style then use your \LaTeX\ system's package installer to add it. If this is not possible you can also revert to the respective bibliography styles without the ``\texttt{-narrow}'' in the file name.\\[1em]
DVI-based processes to compile the template apparently cannot handle the different font so, by default, the template file uses the \texttt{abbrv-doi} bibliography style but the compiled PDF shows you the effect of the \texttt{abbrv-doi-hyperref-narrow} style.
\end{itemize}

